\documentclass[12pt, a4paper, twoside]{article}
\usepackage[utf8]{inputenc}
\usepackage[cm]{fullpage}
\usepackage[inner=2.6cm,outer=1.3cm]{geometry}
\usepackage{fancyhdr}

\begin{document}

\title{Stop Signal Test}
\author{Cristiano Silva Júnior}
\maketitle

\begin{abstract}
Descrevemos o teste SST e as medidas que se esperamos coletar dele. Contamos como está implementado o teste em si e a sua unidade de análise de dados.

\end{abstract}

\section{Introdução}

O $SST$ é um teste para controle inibitório que se baseia no modelo de corrida de cavalos para a inibição humana. Basicamente, ele diz que a inibição é a "competição" entre dois processos: um que diz para responder ao estímulo, e outro para inibir o estímulo. Dado que uma pessoa foi pedida para responder a um estímulo, ela tem um tempo para decidir se ela deve ou não responde-lo, e se alguma outra informação for adicionada no meio do processo, ela deve ser apresentada a tempo que o sujeito seja capaz de mudar de ideia. o teste $SST$ procura, então, fornecer que medida é esta tal que o sujeito seja capaz de inibir um estímulo já iniciado.

Matematicamente, podemos modelar este processo como uma distribuição normal ao redor do tempo em que a pessoa demorar responder a um estímulo de ida. Isto é, dado que houve um estímulo de ida no momento 0, existe uma probabilidade de que a pessoa responda a este estímulo, e ela gira em torno do tempo médio de resposta. Um estímulo inibitório tem mais chance de suceder caso ele esteja mais próximo do estímulo de ida, mas o quão mais próximo ele deve estar? 

Para medir isto, o teste SST foi concebido. Em nossa versão, em um momento do teste, será apresentada uma seta apontando à esquerda ou à direita ao sujeito. Pode ser ou não que um estímulo sonoro seja apresentado ligeiramente após a apresentação da seta. Caso este não apareça, ele deve responder o teste, indicando para qual direção a seta aponta. Caso contrário, ele deve inibir sua resposta, esperando que a próxima rodada comece.

\subsection{Equações}

Aqui, denotaremos $\langle M \rangle$ como a média da medida $\langle M \rangle$. Com os conceitos do $SST$ em mente, temos então 5 variáveis para calcular: o $RT$; o $SSD$; o $SSRT$; a taxa de inibição ($\%INHIB$) e a porcentagem de ausências ($\%AUS$).

O $\langle RT \rangle$ é o tempo médio de reação do sujeito, e é calculado como sendo a soma dos tempos de reação dividida pelo número total de estímulos de ida. 

O $\langle SSD \rangle$ é o tempo médio de inibição do sujeito, e é calculado como a soma dos tempos de \textit{delay} do sujeito dividido pelo número de vezes que ele conseguiu inibir o sinal com sucesso.

O $SSRT$ é a medida crítica do teste, e é a diferença entre o $\langle RT \rangle$ e o $\langle SSD \rangle$. No nosso entendimento, um $SSRT$ negativo implica que o sujeito falhou o teste.

O $\%INHIB$ é uma medida auxiliar, e indica a porcentagem de sucessos de inibição. No nosso caso, cabe dizer que, como o número de vezes que o sujeito deve inibir sua reação é pequeno, esta medida é bastante sensível.

O $\%AUS$ é uma medida definida aqui no laboratório, e é um indicativo do quão bem o sujeito realizou o teste. É calculada como a porcentagem de vezes que o sujeito deixou de responder a um estímulo de ida.

Assim,

$$\langle RT \rangle = \frac{\sum RT}{totalGo}$$
$$\langle SSD \rangle = \frac{\sum SSD}{correctStop}$$
$$SSRT = \langle RT \rangle - \langle SSD \rangle$$
$$\%INHIB = \frac{correctStop}{totalStop} $$
$$\%AUS = \frac{totalGo-correctGo}{totalGo} $$

\section{Teste}

O teste em si foi implementado em E-PRIME.

\section{Analizador}

O programa analizador dos dados pelo E-PRIME foi escrito em Go. Ele consiste de um único arquivo executável, que processa todos os arquivos $.txt$ do diretório especificado. Caso nenhum diretório seja fornecido, o programa usa o diretório atual. Para arquivo $.txt$ do diretório, o executável calcula as equações para cada variável do teste e popula uma tabela. Após ler todos os arquivos, o programa calcula a média e o desvio-padrão. de cada uma dessas variáveis

A saber, a média $\langle x \rangle$ de um conjunto de medidas $\{x_i\}_{i=1}^{n}$ é 
$$\langle x \rangle = \frac{\sum\limits_{i=1}^{n} x_i}{n} $$
E o desvio padrão $S$ deste mesmo conjunto de medidas é definido como
$$S = \sqrt{\frac{\sum\limits_{i=1}^{n} (\langle x \rangle - x_i)^2}{n}} $$

\subsection{Arquivo de dados gerado}

O arquivo de dados gerado é uma tabela no formato TSV, relacionando cada arquivo estudado; os dados coletados e analisados de cada um deles; e uma análise geral daquele conjunto de dados em questão. Ele possui algumas colunas:

\begin{enumerate}
	\item Nome do arquivo
	\item totalGo (número total de estímulos de ida)
	\item correctGo (número correto de estímulos de ida)
	\item totalStop (número total de estímulos de parada)
	\item correctStop (número correto de estímulos de parada)
	\item $\%INHIB$
	\item $\%AUS $
	\item $\langle SSD \rangle$
	\item $\langle RT \rangle$
	\item $SSRT$
\end{enumerate}

Ao final do processamento, esta tabela contém ainda a média e o desvio-padrão de cada uma das medidas listadas em suas respectivas colunas.

\end{document}